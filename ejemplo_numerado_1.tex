\documentclass{article}
\usepackage[letterpaper]{geometry}
%\usepackage{mwe}%load graphix,lipsum,...
\usepackage{xcolor}
\usepackage{caption}
\usepackage{booktabs}
\usepackage[most]{tcolorbox}
\tcbuselibrary{skins}
\tcbuselibrary{breakable}
\usepackage{lipsum}

%\newcounter{example}[section]
%\renewcommand\theexample{\thesection.\arabic{example}}

\newtcbtheorem[auto counter, number within = section]
{Example}{Ejemplo}{%
 fonttitle=\bfseries\large,
 enhanced,
 skin=beamer,
 breakable,
 pad at break=0mm,
 colback=green!5!white,
 colframe=red!50!yellow,
 width=\linewidth,%
 }{exa}


\begin{document}

\section{Test Section}

\begin{Example}{jjfjfj}
\emph{En al figura se muestran cuatro partículas colocadas en los vértices de un rectángulo de lados $a$ y $b$, con las cargas eléctricas indicadas. Si $a=8.0 A$ y $b=6.0 A$, encontrar el potencial el\'ectrico en el centro del rectangulo. Tomar}
\newline
{\bf{\large{Soluci\'on:}}}
\newline
La soluci\'on es....
\begin{equation}
P(x)=\sum_{0\le i\le m\\0<j<n}P(i, j)
\end{equation}
\lipsum[1]
\end{Example}

\end{document}