\documentclass{report}
\usepackage[absolute]{textpos}
\usepackage{pdfpages}
\usepackage[most]{tcolorbox}
\usepackage{lipsum}
\usepackage{cleveref}
\makeatletter
\crefformat{tcb@cnt@Example}{example~#2#1#3}
\Crefformat{tcb@cnt@Example}{Example~#2#1#3}
\makeatother

\newcounter{test}

\newtcolorbox{test}[2]
[%
breakable,
arc=0pt,
outer arc=0pt,
coltitle=black,
fonttitle=\bfseries,
boxrule=0pt,
colframe=gray!30,
colback=gray!30,
title after break={Example~\thetest\ (Continued)}]{%
before upper={\stepcounter{test}\textbf{Example~\thetest.\ }%
},
label={#2},
#1}

\newtcbtheorem[auto counter, number within = chapter]
{Example}{Ejem\smash{p}lo}{%                                                        
 fonttitle=\bfseries\large,
 title after break={\bfseries{...continuaci\'on del ejemplo ~\thetcbcounter.}},
 enhanced,
 skin=beamer,
 beamer,
 drop fuzzy shadow,
 breakable,
 colback=green!5!white,
 colframe=red!50!yellow,
 width=\linewidth
}{exa}


\newtcbtheorem{corollary}{Corollary}{%
  breakable,
  colback=blue!10,
  colframe=blue!25,
  fonttitle=\bfseries
}{cor}



\colorlet{xlightblue}{blue!5}

\newtcolorbox{beamerlikethm}[1]{
  title=#1,
  beamer,
  colback=xlightblue,
  colframe=blue!30,
  fonttitle=\bfseries,
  left=1mm,
  right=1mm,
  top=1mm,
  bottom=1mm,
  middle=1mm
}


\newtcolorbox{beamerlikethm2}[1]{
  title=#1,
  beamer, 
  colback=xlightblue,
  colframe=blue!30,
  fonttitle=\bfseries,
  left=1mm,
  right=1mm,
  top=1mm,
  bottom=1mm,
  middle=1mm,
  breakable,
}

\begin{document}
\chapter{adadad}
\section{Seccion}
\begin{Example}{Cargas en un cuadrado}{exam}
 \emph{En al figura se muestran cuatro partículas colocadas en los vértices de un rectángulo de lados $a$ y $b$, con las cargas eléctricas indicadas. Si $a=8.0 A$ y $b=6.0 A$, encontrar el potencial el\'ectrico en el centro del rectangulo. Tomar}
\newline
{\bf{\large{Soluci\'on:}}}
\newline
La soluci\'on es....
\begin{equation}
P(x)=\sum_{0\le i\le m\\0<j<n}P(i, j)
\end{equation}

\lipsum[1-6]
\end{Example}
\begin{Example}{Cargas en un cuadrado}{exam}
 \emph{En al figura se muestran cuatro partículas colocadas en los vértices de un rectángulo de lados $a$ y $b$, con las cargas eléctricas indicadas. Si $a=8.0 A$ y $b=6.0 A$, encontrar el potencial el\'ectrico en el centro del rectangulo. Tomar}
\newline
{\bf{\large{Soluci\'on:}}}
\newline
La soluci\'on es....
\begin{equation}
P(x)=\sum_{0\le i\le m\\0<j<n}P(i, j)
\end{equation}

\lipsum[1-6]
\end{Example}
\noindent
\Cref{exa:exam} and~\cref{exa:exam}

\begin{test}{testa}
\lipsum[1-6]
\end{test}
\begin{tcolorbox}
[enhanced,breakable,pad at break=0mm,colback=green!5!,colframe=green!35!black,title=\large\bf{Trabajo}]
\begin{equation}
U=\frac{1}{2}CV^{2}=\frac{Q^{2}}{2C}
\end{equation}
\end{tcolorbox}

\begin{corollary}{Title}{dummy}
    \[
      x+y=y+x
    \]
  \end{corollary}
  
  \begin{beamerlikethm}{Theorem (Pythagoras)}
\[ a^2 + b^2 = c^2 \]
\end{beamerlikethm}

\begin{beamerlikethm2}{Theorem (Pythagoras)}
\[ a^2 + b^2 = c^2 \]
\end{beamerlikethm2}



\begin{beamerlikethm2}{Something}
\lipsum[5]
\begin{equation}
U=\frac{1}{2}CV^{2}=\frac{Q^{2}}{2C}
\end{equation}
\end{beamerlikethm2}
\newpage


%\begin{textblock*}{297mm}(0mm,0mm)
%        \includegraphics[width=\paperwidth]{C:/ebookF/fig_general/92283026}
%\end{textblock*}
\newpage
\thispagestyle{empty}
\begin{textblock*}{\paperwidth}(0mm,0mm)
   \noindent\includegraphics[width=\paperwidth,height=\paperheight]{C:/ebookF/fig_general/2}
\end{textblock*}
\mbox{}
\newpage
\end{document}