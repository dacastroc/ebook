%2multibyte Version: 5.50.0.2960 CodePage: 65001
\documentclass[notes=show]{beamer}%
\usepackage{mathpazo}
\usepackage{hyperref}
\usepackage{multimedia}%
\usepackage{amsmath}%
\setcounter{MaxMatrixCols}{30}%
\usepackage{amsfonts}%
\usepackage{amssymb}%
\usepackage{graphicx}
%TCIDATA{OutputFilter=latex2.dll}
%TCIDATA{Version=5.50.0.2960}
%TCIDATA{Codepage=65001}
%TCIDATA{CSTFile=beamer.cst}
%TCIDATA{Created=Sunday, October 06, 2013 17:31:05}
%TCIDATA{LastRevised=Sunday, October 06, 2013 17:33:28}
%TCIDATA{<META NAME="GraphicsSave" CONTENT="32">}
%TCIDATA{<META NAME="SaveForMode" CONTENT="2">}
%TCIDATA{BibliographyScheme=Manual}
%TCIDATA{<META NAME="DocumentShell" CONTENT="Other Documents\SW\Slides - Beamer">}
%BeginMSIPreambleData
\providecommand{\U}[1]{\protect\rule{.1in}{.1in}}
%EndMSIPreambleData
\newenvironment{stepenumerate}{\begin{enumerate}[<+->]}{\end{enumerate}}
\newenvironment{stepitemize}{\begin{itemize}[<+->]}{\end{itemize} }
\newenvironment{stepenumeratewithalert}{\begin{enumerate}[<+-| alert@+>]}{\end{enumerate}}
\newenvironment{stepitemizewithalert}{\begin{itemize}[<+-| alert@+>]}{\end{itemize} }
\usetheme{Madrid}
\usepackage{lmodern}
\usepackage{tikz}
\usepackage[utf8]{inputenc}
\usepackage[ngerman]{babel}
\usepackage[T1]{fontenc}
\usepackage[most]{tcolorbox}
\setbeamertemplate{blocks}[rounded][shadow=true]
\setbeamercolor{block title}{fg=black,bg=black!70}
\setbeamercolor{block body}{fg=white,bg=black}
\newtcolorbox{myblock}[2][]{beamer,title=#2,fonttitle=\sffamily,
left=1mm,right=1mm,top=1mm,bottom=1mm,arc=2mm,
colback=black,colupper=white,colframe=yellow,
coltitle=black,title style={top color=red!70,bottom color=yellow},
#1}
%BeginMSIPreambleData
\ifx\pdfoutput\relax\let\pdfoutput=\undefined\fi
\newcount\msipdfoutput
\ifx\pdfoutput\undefined\else
\ifcase\pdfoutput\else
\msipdfoutput=1
\ifx\paperwidth\undefined\else
\ifdim\paperheight=0pt\relax\else\pdfpageheight\paperheight\fi
\ifdim\paperwidth=0pt\relax\else\pdfpagewidth\paperwidth\fi
\fi\fi\fi
%EndMSIPreambleData
\begin{document}

\title{Creating Beamer presentations in \textsl{Scientific WorkPlace\/}{} and
\textsl{Scientific Word}{}}
\subtitle{Impressive slide presentations}
\author{MacKichan Software Technical Support}
\institute{Delete or rename Institute field}
\date{[01/07]January 2007}
\maketitle

\section{Creating Beamer presentations in Scientific WorkPlace and Scientific
Word}

\subsection{Slides - Beamer}%

%TCIMACRO{\TeXButton{BeginFrame}{\begin{frame}}}%
%BeginExpansion
\begin{frame}%
%EndExpansion
%

%TCIMACRO{\QTR{frametitle}{Slides - Beamer}}%
%BeginExpansion
\frametitle{Slides - Beamer}%
%EndExpansion
%

%TCIMACRO{\TeXButton{block}{\begin{block}%
%{Title which background color i like to shade}
%Block entry
%\end{block}
%\begin{myblock}{Title which background color i like to shade}
%Block entry
%\end{myblock}}}%
%BeginExpansion
\begin{block}{Title which background color i like to shade}
Block entry
\end{block}
\begin{myblock}{Title which background color i like to shade}
Block entry
\end{myblock}%
%EndExpansion


\begin{itemize}
\item The class provides

\begin{itemize}
\item Control of layout, color, and fonts

\item A variety of list and list display mechanisms

\item Dynamic transitions between slides

\item Presentations containing text, mathematics, graphics, and animations
\end{itemize}

\item A single document file contains an entire Beamer presentation.

\item Each slide in the presentation is created inside a frame environment.

\item To produce a sample presentation in \textsl{SWP }or \textsl{SW}, typeset
this shell document with \textsc{pdf}%
%TCIMACRO{\TeXButton{LaTeX}{\LaTeX{}} }%
%BeginExpansion
\LaTeX{}
%EndExpansion
.
\end{itemize}%

%TCIMACRO{\TeXButton{EndFrame}{\end{frame}}}%
%BeginExpansion
\end{frame}%
%EndExpansion


\subsection{Beamer Files}%

%TCIMACRO{\TeXButton{BeginFrame}{\begin{frame}}}%
%BeginExpansion
\begin{frame}%
%EndExpansion
%

%TCIMACRO{\QTR{frametitle}{Beamer Files}}%
%BeginExpansion
\frametitle{Beamer Files}%
%EndExpansion


\begin{itemize}
\item The document class base file for this shell is \texttt{beamer.cls}.

\item To see the available class options, choose \textsf{Typeset, }choose
\textsf{Options and Packages}, select the \textsf{Class Options} tab, and then
click the \textsf{Modify} button.

\item This shell specifies showing all notes but otherwise uses the default
class options.

\item The typesetting specification for this shell document uses these options
and packages with the defaults indicated:
\end{itemize}

\begin{center}
\
\begin{tabular}
[c]{ll}%
\textbf{Options and Packages} & \textbf{Defaults}\\\hline
Document class options & Show notes\\
Packages: & \\
\quad hyperref & Standard\\
\quad mathpazo & None\\
\quad multimedia & None\\\hline
\end{tabular}

\end{center}%

%TCIMACRO{\TeXButton{EndFrame}{\end{frame}}}%
%BeginExpansion
\end{frame}%
%EndExpansion


\subsection{Using This Shell}%

%TCIMACRO{\TeXButton{BeginFrame}{\begin{frame}}}%
%BeginExpansion
\begin{frame}%
%EndExpansion
%

%TCIMACRO{\QTR{frametitle}{Using This Shell}}%
%BeginExpansion
\frametitle{Using This Shell}%
%EndExpansion


\begin{itemize}
\item The front matter of this shell has a number of sample entries that you
should replace with your own.

\item Replace the body of this document with your own text. To start with a
blank document, delete all of the text in this document.

\item Changes to the typeset format of this shell and its associated \LaTeX{}
formatting file (\texttt{beamer.cls}) are not supported by MacKichan Software,
Inc. If you want to make such changes, please consult the \LaTeX{} manuals or
a local \LaTeX{} expert.

\item If you modify this document and export it as \textquotedblleft Slides -
Beamer.shl\textquotedblright\ in the \texttt{Shells%
%TCIMACRO{\TEXTsymbol{\backslash}}%
%BeginExpansion
$\backslash$%
%EndExpansion
Other%
%TCIMACRO{\TEXTsymbol{\backslash}}%
%BeginExpansion
$\backslash$%
%EndExpansion
SW} directory, it will become your new Slides - Beamer shell.
\end{itemize}%

%TCIMACRO{\TeXButton{EndFrame}{\end{frame}}}%
%BeginExpansion
\end{frame}%
%EndExpansion


\subsection{What is Beamer?}%

%TCIMACRO{\TeXButton{BeginFrame}{\begin{frame}}}%
%BeginExpansion
\begin{frame}%
%EndExpansion
%

%TCIMACRO{\QTR{frametitle}{What is Beamer?}}%
%BeginExpansion
\frametitle{What is Beamer?}%
%EndExpansion


\begin{itemize}
\item Beamer is a
%TCIMACRO{\TeXButton{LaTeX}{\LaTeX{}} }%
%BeginExpansion
\LaTeX{}
%EndExpansion
document class that produces beautiful \textsc{pdf}%
%TCIMACRO{\TeXButton{LaTeX}{\LaTeX{}} }%
%BeginExpansion
\LaTeX{}
%EndExpansion
presentations and transparency slides.

\item Beamer presentations feature

\begin{stepitemize}
\item \textsc{pdf}%
%TCIMACRO{\TeXButton{LaTeX}{\LaTeX{}} }%
%BeginExpansion
\LaTeX{}
%EndExpansion
output

\item Global and local control of layout, color, and fonts

\item List items that can appear one at a time

\item Overlays and dynamic transitions between slides

\item Standard
%TCIMACRO{\TeXButton{LaTeX}{\LaTeX{}} }%
%BeginExpansion
\LaTeX{}
%EndExpansion
constructs

\item Typeset text, mathematics $\frac{-b\pm\sqrt{b^{2}-4ac}}{2a}$, and graphics
\end{stepitemize}

\item To produce a sample presentation in \textsl{SWP }or \textsl{SW}, typeset
this shell document with \textsc{pdf}%
%TCIMACRO{\TeXButton{LaTeX}{\LaTeX{}} }%
%BeginExpansion
\LaTeX{}
%EndExpansion
.
\end{itemize}%

%TCIMACRO{\TeXButton{EndFrame}{\end{frame}}}%
%BeginExpansion
\end{frame}%
%EndExpansion


\subsection{Creating frames}%

%TCIMACRO{\TeXButton{BeginFrame}{\begin{frame}}}%
%BeginExpansion
\begin{frame}%
%EndExpansion
%

%TCIMACRO{\QTR{frametitle}{Creating frames}}%
%BeginExpansion
\frametitle{Creating frames}%
%EndExpansion


\begin{stepitemize}
\item All the information in a Beamer\emph{\ }presentation is contained in
\textit{frames.}

\item Each frame corresponds to a single presentation slide.

\item To create frames in a Beamer document,

\begin{stepenumerate}
\item Apply a frame fragment:

\begin{stepitemize}
\item The \textbf{Frame with title and subtitle} fragment starts and ends a
new frame and includes a title and subtitle.

\item The \textbf{Frame with title }fragment starts and ends a new frame and
includes a title.

\item The \textbf{Frame} fragment starts and ends a new frame.
\end{stepitemize}

\item Place the text for the frame between the BeginFrame and EndFrame fields.

\item Enter the frame title and subtitle.

If you used the Frame fragment, apply the Frame title and Frame subtitle text
tags as necessary.
\end{stepenumerate}
\end{stepitemize}%

%TCIMACRO{\TeXButton{EndFrame}{\end{frame}}}%
%BeginExpansion
\end{frame}%
%EndExpansion


\subsection{Learn more about Beamer}%

%TCIMACRO{\TeXButton{BeginFrame}{\begin{frame}}}%
%BeginExpansion
\begin{frame}%
%EndExpansion
%

%TCIMACRO{\QTR{frametitle}{Learn more about Beamer}}%
%BeginExpansion
\frametitle{Learn more about Beamer}%
%EndExpansion


\begin{itemize}
\item This shell and the associated fragments provide basic support for Beamer
in \textsl{SWP }and \textsl{SW}.

\item To learn more about Beamer, see SWSamples/PackageSample-beamer.tex in
your program installation.

\item For complete information, read the BeamerUserGuide.pdf manual found via
a link at the end of SWSamples/PackageSample-beamer.tex.

\item For support, contact \textbf{support@mackichan.com}.
\end{itemize}%

%TCIMACRO{\TeXButton{EndFrame}{\end{frame}}}%
%BeginExpansion
\end{frame}%
%EndExpansion



\end{document}