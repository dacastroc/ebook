%2multibyte Version: 5.50.0.2960 CodePage: 65001
\documentclass{book}%
\usepackage{amsfonts}
\usepackage{amsmath}%
\setcounter{MaxMatrixCols}{30}%
\usepackage{amssymb}%
\usepackage{graphicx}
%TCIDATA{OutputFilter=latex2.dll}
%TCIDATA{Version=5.50.0.2960}
%TCIDATA{Codepage=65001}
%TCIDATA{CSTFile=40 LaTeX Book.cst}
%TCIDATA{Created=Tuesday, October 01, 2013 15:03:05}
%TCIDATA{LastRevised=Tuesday, October 01, 2013 15:22:06}
%TCIDATA{<META NAME="GraphicsSave" CONTENT="32">}
%TCIDATA{<META NAME="SaveForMode" CONTENT="1">}
%TCIDATA{BibliographyScheme=Manual}
%TCIDATA{<META NAME="DocumentShell" CONTENT="Standard LaTeX\Standard LaTeX Book">}
%BeginMSIPreambleData
\providecommand{\U}[1]{\protect\rule{.1in}{.1in}}
%EndMSIPreambleData
\newtheorem{theorem}{Theorem}
\newtheorem{acknowledgement}[theorem]{Acknowledgement}
\newtheorem{algorithm}[theorem]{Algorithm}
\newtheorem{axiom}[theorem]{Axiom}
\newtheorem{case}[theorem]{Case}
\newtheorem{claim}[theorem]{Claim}
\newtheorem{conclusion}[theorem]{Conclusion}
\newtheorem{condition}[theorem]{Condition}
\newtheorem{conjecture}[theorem]{Conjecture}
\newtheorem{corollary}[theorem]{Corollary}
\newtheorem{criterion}[theorem]{Criterion}
\newtheorem{definition}[theorem]{Definition}
%\newtheorem{example}[theorem]{Example}
\newtheorem{exercise}[theorem]{Exercise}
\newtheorem{lemma}[theorem]{Lemma}
\newtheorem{notation}[theorem]{Notation}
\newtheorem{problem}[theorem]{Problem}
\newtheorem{proposition}[theorem]{Proposition}
\newtheorem{remark}[theorem]{Remark}
\newtheorem{solution}[theorem]{Solution}
\newtheorem{summary}[theorem]{Summary}
\newenvironment{proof}[1][Proof]{\noindent\textbf{#1.} }{\ \rule{0.5em}{0.5em}}
\usepackage{xcolor}
\usepackage{amsthm}
\usepackage{amssymb}
\usepackage{thmtools}
\usepackage{lipsum}
\colorlet{myred}{red!80!black}
\declaretheoremstyle[
spaceabove=\topsep, spacebelow=\topsep,
headfont=\normalfont\color{myred},
notefont=\mdseries\color{myred}, notebraces={(}{)},
bodyfont=\normalfont,
postheadspace=\newline,
headpunct=,
numberwithin=chapter,
postheadhook=\leavevmode%
\interlinepenalty 10000%
\vskip-1.3\baselineskip%
\noindent{\color{myred}\rule{\textwidth}{1pt}}%
\interlinepenalty 10000%
\vskip0.3\baselineskip\noindent,
qed=\textcolor{myred}{$\blacksquare$}
]{mystyle}
\declaretheorem[style=mystyle]{example}
\renewcommand\theexample{\thechapter-\arabic{example}}
\begin{document}

\frontmatter
\title{Standard
%TCIMACRO{\TeXButton{LaTeX}{\LaTeX{}} }%
%BeginExpansion
\LaTeX{}
%EndExpansion
Book}
\author{The Author}
\date{The Date}
\maketitle
\tableofcontents

\chapter*{Preface}

\markboth{PREFACE}{PREFACE}This is the preface. It is an unnumbered chapter.
The [markboth]
%TCIMACRO{\TeXButton{TeX}{\TeX{}} }%
%BeginExpansion
\TeX{}
%EndExpansion
field at the beginning of this paragraph sets the correct page heading for the
Preface portion of the document. The preface does not appear in the table of contents.

\chapter{Introduction}

The introduction is entered using the usual chapter tag. Since the
introduction chapter appears before the [mainmatter]
%TCIMACRO{\TeXButton{TeX}{\TeX{}} }%
%BeginExpansion
\TeX{}
%EndExpansion
field, it is an unnumbered chapter. The primary difference between the preface
and the introduction in this shell document is that the introduction will
appear in the table of contents and the page headings for the introduction are
automatically handled without the need for the [markboth]
%TCIMACRO{\TeXButton{TeX}{\TeX{}} }%
%BeginExpansion
\TeX{}
%EndExpansion
field. You may use either or both methods to create chapters at the beginning
of your document. You may also delete these preliminary chapters.

\mainmatter


\part{The First Part}

\chapter{Standard
%TCIMACRO{\TeXButton{LaTeX}{\LaTeX{}} }%
%BeginExpansion
\LaTeX{}
%EndExpansion
Book}

This document illustrates the appearance of a book created with the shell
\textbf{Standard LaTeX Book}. The shell automatically adds blank pages after
the title page, the table of contents, the preface, and where necessary to
ensure that new chapters begin on odd-numbered pages. The shell doesn't
contain an abstract. Blank pages carry headers and page numbers.

The standard \LaTeX{} shells provide the most general and portable set of
document features. You can achieve almost any typesetting effect by beginning
with a standard shell and adding \LaTeX{} packages as necessary.

The document class base file for this shell is \texttt{book.cls}. This
typesetting specification supports a number of class options. To see the
available class options, choose \textsf{Typeset, }choose \textsf{Options and
Packages}, select the \textsf{Class Options} tab, and then click the
\textsf{Modify} button. This shell uses the default class options.

The typesetting specification for this shell document uses these options and
packages with the defaults indicated:

\begin{center}
\
\begin{tabular}
[c]{ll}%
\textbf{Options and Packages} & \textbf{Defaults}\\\hline
Document class options & Standard\\
Packages: & \\
\quad amsfonts & None\\
\quad amsmath & Standard\\\hline
\end{tabular}

\end{center}%

%TCIMACRO{\TeXButton{%
%\begin{example}%
%}{\begin{example}}}%
%BeginExpansion
\begin{example}%
%EndExpansion


Let $H$ be a Hilbert space, $C$ be a closed bounded convex subset of $H$, $T$
a nonexpansive self map of $C$. Suppose that as $n\rightarrow\infty$,
$a_{n,k}\rightarrow0$ for each $k$, and $\gamma_{n}=\sum_{k=0}^{\infty}\left(
a_{n,k+1}-a_{n,k}\right)  ^{+}\rightarrow0$. Then for each $x$ in $C$,
$A_{n}x=\sum_{k=0}^{\infty}a_{n,k}T^{k}x$ converges weakly to a fixed point of
$T$ .

The numbered equation
\begin{equation}
u_{tt}-\Delta u+u^{5}+u\left\vert u\right\vert ^{p-2}=0\text{ in }%
\mathbf{R}^{3}\times\left[  0,\infty\right[  \label{eqn1}%
\end{equation}
is automatically numbered as equation \ref{eqn1}.%
%TCIMACRO{\FRAME{ftbphFU}{2.5547in}{2.757in}{0pt}{\Qcb{una figura}}%
%{}{27_figure17a-i.jpg}{\special{ language "Scientific Word";  type "GRAPHIC";
%maintain-aspect-ratio TRUE;  display "USEDEF";  valid_file "F";
%width 2.5547in;  height 2.757in;  depth 0pt;  original-width 6.8225in;
%original-height 7.3647in;  cropleft "0";  croptop "1";  cropright "1";
%cropbottom "0";
%filename '../../../../Gauss/Examenes/Electricidad/27_Figure17a-I.jpg';file-properties "XNPEU";}%
%}}%
%BeginExpansion
\begin{figure}[ptbh]%
\centering
\includegraphics[
natheight=7.364700in,
natwidth=6.822500in,
height=2.757in,
width=2.5547in
]%
{../../../../Gauss/Examenes/Electricidad/27_Figure17a-I.jpg}%
\caption{una figura}%
\end{figure}
%EndExpansion


\textit{El resultado anterior se puede utilizar para obtener el campo
el\'{e}ctrico en un punto sobre un eje que pasa por el centro del disco y es
perpendicular al plano del mismo. Para ello, en la ecuaci\'{o}n (\ref{2.4.2}),
se hace }$a=0$\textit{ y }$b=R$\textit{, resultando que}%

\[
V=2\pi k\sigma\left[  \sqrt{R^{2}+z^{2}}-z.\right]
\]
\textit{El campo del disco se calcula como sigue:}%

\begin{equation}
\mathbf{E}=-\frac{\partial V}{\partial z}\mathbf{k=}2\pi k\sigma\left[
1-\frac{z}{\left[  R^{2}+z^{2}\right]  ^{1/2}}\right]  \mathbf{k}
\label{2.4.3}%
\end{equation}
\textit{Mediante la ecuaci\'{o}n (\ref{2.4.3}) se puede obtener el campo
el\'{e}ctrico de un plano no conductor que tiene una densidad superficial de
carga uniforme, }$\sigma.$\textit{ Para tal fin podemos considerar que un
plano es un disco cuyo radio tienede a infinito. Entonces, haciendo que }%
$R$\textit{ tienda a infinito en la ecuaci\'{o}n (2.4.3), el segundo
t\'{e}rmino del miembro deredcho de esta acuaci\'{o}n tienede a cero y resulta
que el campo del plano es }%
\begin{equation}
\mathbf{E}=-\frac{\partial V}{\partial z}\mathbf{k=}2\pi k\sigma
\mathbf{k=}\frac{\sigma}{2\epsilon_{o}} \label{2.4.4}%
\end{equation}


\textit{\noindent Este campo es independiente de las coordenadas, es
perpendicular al plano se dice que es uniforme. Dicho campo se representa por
l\'{\i}neas que se dirigen perpendicularmente hacie el plano si }$\sigma
$\textit{ es negativa y en direcci\'{o}n opueta si }$\sigma$\textit{ es
positiva. }%

%TCIMACRO{\TeXButton{%
%\end{example}%
%}{\end{example}}}%
%BeginExpansion
\end{example}%
%EndExpansion
%

%TCIMACRO{\TeXButton{%
%\begin{example}%
%}{\begin{example}}}%
%BeginExpansion
\begin{example}%
%EndExpansion
%

%TCIMACRO{\TEXTsymbol{\backslash}}%
%BeginExpansion
$\backslash$%
%EndExpansion
lipsum[2-3]%

%TCIMACRO{\TeXButton{%
%\end{example}%
%}{\end{example}}}%
%BeginExpansion
\end{example}%
%EndExpansion


\chapter{Using This Shell}

The front matter of this shell has a number of sample entries that you should
replace with your own. Replace the body of this document with your own text.
To start with a blank document, you may delete the preliminary chapters and
the text in this document. Do not delete the [mainmatter]
%TCIMACRO{\TeXButton{TeX}{\TeX{}} }%
%BeginExpansion
\TeX{}
%EndExpansion
field found above in a paragraph by itself or the numbering of different
objects will be wrong.

Changes to the typeset format of this shell and its associated \LaTeX{}
formatting file (\texttt{book.cls}) are not supported by MacKichan Software,
Inc. If you want to make such changes, please consult the \LaTeX{} manuals or
a local \LaTeX{} expert.

If you modify this document and export it as \textquotedblleft Standard LaTeX
Book.shl\textquotedblright\ in the \texttt{Shells%
%TCIMACRO{\TEXTsymbol{\backslash}}%
%BeginExpansion
$\backslash$%
%EndExpansion
Standard LaTeX} directory, it will become your new Standard LaTeX Book style shell.

\chapter{Features of This Shell}

\section{Section Heading}

Use the Section tag for major sections, and the Subsection tag for subsections.

\subsection{Subsection}

This is just some harmless text under a subsection.

\subsubsection{Subsubsection}

This is just some harmless text under a subsubsection.

\paragraph{Subsubsubsection}

This is just some harmless text under a subsubsubsection.

\subparagraph{Subsubsubsubsection}

This is just some harmless text under a subsubsubsubsection.

\section{Tags}

You can apply the logical markup tag \emph{Emphasized}.

You can apply the visual markup tags \textbf{Bold}, \textit{Italics},
\textrm{Roman}, \textsf{Sans Serif}, \textsl{Slanted}, \textsc{Small Caps},
and \texttt{Typewriter}.

You can apply the special mathematics-only tags $\mathbb{BLACKBOARD}$
$\mathbb{BOLD}$, $\mathcal{CALLIGRAPHIC}$, and $\mathfrak{fraktur}$. Note that
blackboard bold and calligraphic are correct only when applied to uppercase
letters A through Z.

You can apply the size tags {\tiny tiny}, {\scriptsize scriptsize},
{\footnotesize footnotesize}, {\small small}, {\normalsize normalsize},
{\large large}, {\Large Large}, {\LARGE LARGE}, {\huge huge} and {\Huge Huge}.

This is a Body Math paragraph. Each time you press the Enter key, Scientific
WorkPlace switches to mathematics mode. This is convenient for carrying out
``scratchpad'' computations.

Following is a group of paragraphs marked as Short Quote. This environment is
appropriate for a short quotation or a sequence of short quotations.

\begin{quote}
The only thing we have to fear is fear itself. \emph{Franklin D. Roosevelt,
}Mar. 4, 1933

Ask not what your country can do for you; ask what you can do for your
country. \emph{John F. Kennedy, }Jan. 20. 1961

There is nothing wrong with America that cannot be cured by what is right with
America. \emph{William J. \textquotedblleft Bill\textquotedblright\ Clinton,
}Jan. 21, 1993
\end{quote}

The Long Quotation tag is used for quotations of more than one paragraph.
Following is the beginning of \emph{Alice's Adventures in Wonderland }by Lewis Carroll:

\begin{quotation}
Alice was beginning to get very tired of sitting by her sister on the bank,
and of having nothing to do: once or twice she had peeped into the book her
sister was reading, but it had no pictures or conversations in it, `and what
is the use of a book,' thought Alice `without pictures or conversation?'

So she was considering in her own mind (as well as she could, for the hot day
made her feel very sleepy and stupid), whether the pleasure of making a
daisy-chain would be worth the trouble of getting up and picking the daisies,
when suddenly a White Rabbit with pink eyes ran close by her.

There was nothing so very remarkable in that; nor did Alice think it so very
much out of the way to hear the Rabbit say to itself, `Oh dear! Oh dear! I
shall be late!' (when she thought it over afterwards, it occurred to her that
she ought to have wondered at this, but at the time it all seemed quite
natural); but when the Rabbit actually took a watch out of its
waistcoat-pocket, and looked at it, and then hurried on, Alice started to her
feet, for it flashed across her mind that she had never before seen a rabbit
with either a waistcoat-pocket, or a watch to take out of it, and burning with
curiosity, she ran across the field after it, and fortunately was just in time
to see it pop down a large rabbit-hole under the hedge.

In another moment down went Alice after it, never once considering how in the
world she was to get out again.
\end{quotation}

Use the Verbatim tag when you want \LaTeX{} to preserve spacing, perhaps when
including a fragment from a program such as:
\begin{verbatim}
#include <iostream>        // < > is used for standard libraries.
void main(void)            // "main" method always called first.
{
  cout << "Hello World.";  // Send to output stream.
}
\end{verbatim}

\section{Mathematics and Text}

Let $H$ be a Hilbert space, $C$ be a closed bounded convex subset of $H$, $T$
a nonexpansive self map of $C$. Suppose that as $n\rightarrow\infty$,
$a_{n,k}\rightarrow0$ for each $k$, and $\gamma_{n}=\sum_{k=0}^{\infty}\left(
a_{n,k+1}-a_{n,k}\right)  ^{+}\rightarrow0$. Then for each $x$ in $C$,
$A_{n}x=\sum_{k=0}^{\infty}a_{n,k}T^{k}x$ converges weakly to a fixed point of
$T$ .

The numbered equation
\begin{equation}
u_{tt}-\Delta u+u^{5}+u\left|  u\right|  ^{p-2}=0\text{ in }\mathbf{R}%
^{3}\times\left[  0,\infty\right[ \label{eqn1}%
\end{equation}
is automatically numbered as equation \ref{eqn1}.

\section{Lists Environments}

You can create numbered, bulleted, and description lists using the Item Tag
popup list on the Tag toolbar.

\begin{enumerate}
\item List item 1

\item List item 2

\begin{enumerate}
\item A list item under a list item.

The typeset style for this level is different than the screen style. The
screen shows a lower case alphabetic character followed by a period while the
typeset style uses a lower case alphabetic character surrounded by parentheses.

\item Just another list item under a list item.

\begin{enumerate}
\item Third level list item under a list item.

\begin{enumerate}
\item Fourth and final level of list items allowed.
\end{enumerate}
\end{enumerate}
\end{enumerate}
\end{enumerate}

\begin{itemize}
\item Bullet item 1

\item Bullet item 2

\begin{itemize}
\item Second level bullet item.

\begin{itemize}
\item Third level bullet item.

\begin{itemize}
\item Fourth (and final) level bullet item.
\end{itemize}
\end{itemize}
\end{itemize}
\end{itemize}

\begin{description}
\item[Description List] Each description list item has a term followed by the
description of that term. Double click the term box to enter the term, or to
change it.

\item[Bunyip] Mythical beast of Australian Aboriginal legends.
\end{description}

\section{Theorem-Like Environments}

The following theorem-like environments (in alphabetical order) are available
in this style.

\begin{acknowledgement}
This is an acknowledgement
\end{acknowledgement}

\begin{algorithm}
This is an algorithm
\end{algorithm}

\begin{axiom}
This is an axiom
\end{axiom}

\begin{case}
This is a case
\end{case}

\begin{claim}
This is a claim
\end{claim}

\begin{conclusion}
This is a conclusion
\end{conclusion}

\begin{condition}
This is a condition
\end{condition}

\begin{conjecture}
This is a conjecture
\end{conjecture}

\begin{corollary}
This is a corollary
\end{corollary}

\begin{criterion}
This is a criterion
\end{criterion}

\begin{definition}
This is a definition
\end{definition}

\begin{example}
This is an example
\end{example}

\begin{exercise}
This is an exercise
\end{exercise}

\begin{lemma}
This is a lemma
\end{lemma}

\begin{proof}
This is the proof of the lemma.
\end{proof}

\begin{notation}
This is notation
\end{notation}

\begin{problem}
This is a problem
\end{problem}

\begin{proposition}
This is a proposition
\end{proposition}

\begin{remark}
This is a remark
\end{remark}

\begin{summary}
This is a summary
\end{summary}

\begin{theorem}
This is a theorem
\end{theorem}

\begin{proof}
[Proof of the Main Theorem]This is the proof.
\end{proof}

\appendix


\chapter{The First Appendix}

The appendix fragment is used only once. Subsequent appendices can be created
using the Chapter Section/Body Tag.

\backmatter


\chapter{Afterword}

The back matter often includes one or more of an index, an afterword,
acknowledgements, a bibliography, a colophon, or any other similar item. In
the back matter, chapters do not produce a chapter number, but they are
entered in the table of contents. If you are not using anything in the back
matter, you can delete the back matter
%TCIMACRO{\TeXButton{TeX}{\TeX{}} }%
%BeginExpansion
\TeX{}
%EndExpansion
field and everything that follows it.
\end{document}